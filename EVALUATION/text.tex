\part{Evaluation}
	\chapter{Introduction}
	The Evaluation chapter aims to determine the strengths and weaknesses of the project. The choice of equipment and techniques will be evaluated, taking into account their original reasons for being employed and how useful they were during the project's development. In addition the strengths and weaknesses of the project's approach will be evaluated. Situations where the project's approach allowed for progress to be made or obstacles overcome will be described, as well as incidents where something different may have yielded more success. Finally the product itself will be evaluated. The original project requirements outlined in the analysis will be touched on again, and then the matter of whether or not they have been satisfied will be discussed. Attention will be given to aspects such as the original reasoning behind the objectives, the preliminary research (or possibly lack thereof) that contributes to the objectives success/failure as well as any strengths or faults during the project's synthesis. Finally, some conclusions on the project will be reached followed by recommendations for further work in the same vein. 
	
	\chapter{Process Evaluation}
	%fitness for purpose and build quality sections from tor
	
	%talk about obvious issues with attempting to implement the SDK - get the actual linux error message(s)?
	%gives examples of previously attempted solutions and code snippets, e.g
	%i tried implementing manual getc and putc *actual code*
	%i tried using a byte lookup table on this *actual code*
	%i tried implementing a non-blocking interrupt handler attached to the rx channel *actual code*
	The way in which the project was approached had some strengths. The preliminary research into the relevant project fields prior to development helped in understanding what it was that precisely needed to be done to achieve the project goals.
	
	One of the biggest issues in the project was a grossly inefficient usage of time. The obstacle of establishing communication between the LIDAR sensor and the microcontroller came up right near the start of the project, but it was a great length of time before any actual success in having the two devices interface with eachother was realized. 
	
	\chapter{Product Evaluation}
	
	
	\chapter{Conclusions and Recommendations}