\part{Evaluation}
	\chapter{Introduction}
	The Evaluation chapter aims to determine the strengths and weaknesses of the project. The choice of equipment and techniques will be evaluated, taking into account their original reasons for being employed and how useful they were during the project's development. In addition the strengths and weaknesses of the project's approach will be evaluated. Situations where the project's approach allowed for progress to be made or obstacles overcome will be described, as well as incidents where something different may have yielded more success. Finally the product itself will be evaluated. The original project requirements outlined in the analysis will be touched on again, and then the matter of whether or not they have been satisfied will be discussed. Attention will be given to aspects such as the original reasoning behind the objectives, the preliminary research (or possibly lack thereof) that contributes to the objectives success/failure as well as any strengths or faults during the project's synthesis. Finally, some conclusions on the project will be reached followed by recommendations for further work in the same vein. 
	
	\chapter{Product Evaluation}
	
	
	
	
	\chapter{Process Evaluation}
	\label{evaluation:processevaluation}
	%fitness for purpose and build quality sections from tor
	
	%talk about obvious issues with attempting to implement the SDK - get the actual linux error message(s)?
	%gives examples of previously attempted solutions and code snippets, e.g
	%i tried implementing manual getc and putc *actual code*
	%i tried using a byte lookup table on this *actual code*
	%i tried implementing a non-blocking interrupt handler attached to the rx channel *actual code*
	The way in which the project was approached had some strengths. The preliminary research into the relevant project fields helped in understanding what it was that precisely needed to be done to achieve the project goals. As well, I felt the review of tools and techniques was for the most part very helpful. The Terms of Reference featured some objectives that weren't entirely clear due to a few unknowns, most notably the objective dealing with how viable the drone was for SLAM and what the best way of storing map data would be. The tools and techniques review allowed for the specifics behind what to do here to be understood, as the preliminary investigation into file write speeds and how a radio transmitter would be integrated into the project helped me reach a decision on what to do.
	
	The project's process was not without its faults however. One of the biggest issues in the project's process was a grossly inefficient usage of time. The obstacle of establishing communication between the LIDAR sensor and the microcontroller came up right near the start of the project, but it was a great length of time before any actual success in having the two devices interface with eachother was realized. The significant amount of time trying to get this aspect of the project to work hurt the entire developmental process, and meant that even if everything else went perfectly there was still a lack of time. The problems compounded and caused a great deal of stress, which in turn made work on the project even worse. There are two key ways that this could have been avoided, or at the very least lessened in severity. The first is simply more research prior to the beginning of the project's development. Had more time been spent looking into the LIDAR, perhaps the modified SDK would have been stumbled upon sooner and this entire process could have been avoided. The second solution lies in the somewhat flawed project methodology. I still firmly stand by the prototyping methodology as being suitable for the product's development, but a more pronounced element of agility would have been good. The way prototyping was followed was having parts of functionality slowly iterated towards, but these iterations within themselves could have been approached in an agile fashion. This would have allowed for the prototyping to continue as intended, but would have additionally meant that the objectives were planned out in a more flexible manner so that blockers could be dealt with easier.
	
		\section{Choice of Equipment and Techniques}
	
	
	\chapter{Conclusions and Recommendations}