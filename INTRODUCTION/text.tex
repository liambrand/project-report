\chapter{Introduction}\label{introduction}
	\section{Background}
	This project aims to develop a ground based mobile robotic platform capable of self navigation and mapping. This will allow it to be used for the navigation and exploration of unknown areas.
	
	Exploration and autonomous robotics have went hand in hand for a while now. Programs such as Voyager have resulted in the development of unmanned craft that has set out to try and explore places not humanly accessible so that they may be understood. Unmanned exploration is not restricted to outer space however, with caves\citep{mcfarlane2013integrated} and deep sea sinkholes\citep{carnegie2007sinkhole} being explored with unmanned craft as well. The ability to map environments isn't even specific to exploration either. As the technology behind this has advanced, consumer level products have become available that employ these technologies. Vacuum cleaners are available in stores capable of mapping and navigating around the houses they clean to ensure they have proper coverage of rooms. These autonomous machines have provided an unparalleled level of potential, as they allow for the incredibly accurate measurement of environments whilst also removing aspects such as risks to human life. No longer are these robots things seen in only science fiction. Now you can have one cleaning your living room.
	
	This project aims to develop a product that implements this kind of self-navigational and mapping functionality. A mobile platform will be created that uses range finding technology that enables it to observe its surrounding environment. With these observations, the robot should navigate around areas avoiding obstacles whilst simultaneously mapping the area.
	
	The benefits of such a product are immediately apparently. Regarding exploration, a suitably implemented mobile platform featuring accurate observation capability will be able to provide incredibly accurate data about environments that may be inaccessible, for example the surface of other planets in the solar system or caverns incredibly deep in the ocean. Closer to every day life, self navigational autonomous robots can fulfill purposes such as driverless cars, automated industrial robots and much more.
	
	In order to discover the way in which this will be implemented, several different key aspects will need to be explored. These aspects pertain to the construction of a mobile platform, what hardware the platform will utilise to fulfill its desired functionality (most notably what will be used for observation), and how the observational data that is acquired can be turned into a map.

	\section{Objectives}
	The main objectives of the project are as follows
	\begin{itemize}
		\item Investigate Mobile Robotics - In order to determine how the platform will be created
		\item Investigate Robotic Mapping - To determine how the robot's mapping will be achieved
		\item Propose Requirements - To help provide goals for development
		\item Develop - Build the mobile platform and any appropriate software needed to fulfill requirements
		\item Test - Test the hardware and software components of the developed product
		\item Evaluate - Evaluate the product and the process undertaken to create it
	\end{itemize}

	It's likely that the preliminary research into the mobile robotics and robotic mapping fields will yield multiple potential solutions, but of the solutions that are shown as available only some will be used. These will be determined based on factors such as cost, efficiency and the complexity of implementation.

	\section{Subject of Work}
	The project development took place between September 2018 and April 2019. The created product took the form of a three wheeled robotic chassis, featuring a microcontroller and a LIDAR sensor. The implemented software afforded some very basic movement functionality to the robot, and allowed for the robot to enter a period of scanning followed by writing the obtained scan data to a plugged in Micro-SD card that could be removed and used for map generation. Ultimately the produced product did not satisfy all of the initial requirements, primarily due to issues during development as well as the overall project approach being inadequate and not well thought out enough.
	
	Alternatives with regards to the technology initially meant to be employed here are discussed, with suggestions for improvements to the current product to allow it to better fulfill the project aims as well as ways in which the technology could be taken further.
	
	\section{Plan of Work}
	The project was undertaken with the idea of making use of a prototyping approach. Prototypes would be produced based on requirements gathering, testing, prototyping and observations regarding the effectiveness of previously produced prototypes. Assembly on items such as the physical chassis were done by hand. Software for the microcontroller was developed using the mbed SDK, and was written in C++.
	
	The following subsection will explore each of the core stages that the project went through.
		\subsection{Planning}
		An initial layout regarding the plan of work that needed to be done. Featured in the Terms of Reference, the plan featured a break down of tasks into different deliverables with tasks having milestones and time designations.
		
		\subsection{Analysis}
		Research into relevant project fields was conducted to gain a greater understanding of the fundamentals that would need to be implemented for the project to be a success. This research also aided in determining the correct requirements that the project would need to fulfill to be a success, and these things in turned helped steer the process of choosing what the correct tools and techniques to address the project with were.
		
		\subsection{Synthesis}
		The requirements outlined in the analysis were broken down and a plan was established as to how the chosen tools and techniques could be used to build up a functional implementation to meet these requirements. Following this, the actual implementation took place as well as appropriate testing to determine how well, if it all, the project requirements had been fulfilled by the produced work.
		
		\subsection{Evaluation and Conclusions}
		Finally the product and the process used to approach its creation were evaluated. Following this conclusions were drawn and suggestions for how the technology employed in the project might be taken further were given.
		
		

