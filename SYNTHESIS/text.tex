\part{Synthesis}
	\chapter{Introduction}
	Now that the relevant product fields are understood to a sufficient degree and the product's needs have been established we can begin to plan out how we will achieve the project's objectives. First we will established the methological approach that will structure the product's development. Then, for each of the core product requirements we will look at a high level design overview of how the requirement will be met, the actual implementation of this design, and testing to evaluate how well this requirement has been satisfied by this implementation.
	
	\chapter{Methological Approach}
	The project employed a prototyping methodology for development. First, we must perform an investigation into the system requirements, which was performed in the analysis with a list of specific core requirements being outlined in the conclusion. Following this the first prototype is built, which will resemble an incredibly basic scaled down version of how the final system should ideally look. This prototype is then thoroughly evaluated, with potential changes that would bring the prototype closer to the final system being figured out. Finally, these changes are then implemented and the prototype is evaluated once again. This is repeated until the product has reached its ultimate goals.
	
	%prototyping diagram?
	
	This methodology was employed partially due to the hardware nature of the product. Key features of the robot hinged on aspects like movement and observation being possible, so it was important to get a prototype that has this basic functionality up and working and quickly as possible. The prototyping approach allowed a pure focus on getting these functions working, and once they were the robot could be improved iteratively.
	
		\subsection{Initial Prototype}
		As previously stated, this methodology involves an initial prototype that all future prototypes will be an iteration of. The initial prototype of the project will involve the basic, primary construction of the robot. This will first involve the basic chassis assembly, followed by connecting relevant components to each other and then ensuring that they are all powered. Once this is done, we'll hopefully have an incredibly crude robot that won't be capable of much, but should hopefully serve as a solid foundation for future development toward the project goals.
		
		A similar approach will be followed for processing the map data. It won't be developed simultaneously at first as the robot needs to be capable of storing observational data from the LIDAR before the program can actually do anything, however it will start to be put together once the robot crosses this developmental threshhold.
	
	\chapter{Design}
		\section{Introduction}
		Before implementation of the specific project aims can be implemented, we first need the initial prototype that we can begin iteratively improving. Therefore, it's logical to design the initial robotic prototype first. What follows is an overview of how the initial robot prototype will be designed, followed by a breakdown of each of the individual project aims will be met.
	
		\section{Initial Prototype}
		Here will be the design for the base robot. The aim here is to achieve a solid foundation that all development can build on in order to meet the project objectives.
			\subsection{Hardware}
				\subsubsection{Chassis}
				First off will be the basic assembly of the three wheeled omnidirectional chassis. The chassis is comprised of two triangular metal plates, joined together by metal rods that are screwed into pre drilled holes that line up to eachother. The lower platform is where the motors and mounting points for the omnidirectional wheels are found, so once the plates have been connected the wheels will be pushed onto these mounting points and locked into place.
				% assembly diagram here?
				
				Now it's time for the microcontroller setup. The microcontroller used in this project is the FRDM-K64F, chosen for it's relatively low cost, small form factor and compatibility with the Micro C Operating System which is being employed for the robot's core program. To control the robot we'll also need a motor driver. The dfRobot Quad Motor Driver Shield is being employed for this project, it can be obtained from Amazon for around £13. It was chosen for this relatively low cost, the easily available documentation and the fact that it can easily plug into the FRDM-K64F, the microcontroller of choice for this project. The specifics behind how the shield will be used will be discussed during the movement design section, but in short it allows us to manipulate the energy being sent to the individual wheel motors which will afford us control over how they work and as such control over the robot's movement. Once these two components have been fitted together, the power and ground cables from each of the chassis motors will be plugged into the motor driver shield. Fig \ref{fig:shieldsetup} from the shield's documentation shows an example of this setup with four motors.
				
				\begin{figure}[h]
					\centering
					\includegraphics[width=.7\linewidth]{SYNTHESIS/shield_connection.png}
					\caption{Example shield setup}
					\label{fig:shieldsetup}
				\end{figure}
				
				The assembly shouldn't be overly complex, no soldering should be needed to connect components so it should be mostly a case of fitting things together.
				% explain LIDAR later.... :(
				
				\subsubsection{Power}
				The hardware components that will need powering are the three motors used to drive the omnidirectional wheels, the microcontroller, and the LIDAR sensor. Table \ref{table:1} has a run down of these hardware components, the voltages, and the milliamp consumption that they have.
				
				\begin{table}[h!]
					\centering
					\begin{tabular}{|| l | l | l ||} 
						\hline
						Component & Voltage & Milliamp \\ [0.5ex] 
						\hline
						3x DC Coreless Motor  & 12V & Up to 1400 mA  \\ 
						FRDM-K64F  & 1.71 to 3.6V &  50 mA \\
						RPLIDAR  & 5 to 10V & Up to 1050 mA\\
						\hline
					\end{tabular}
					\caption{Components and respective voltages}
					\label{table:1}
				\end{table}
				
				To save on cost and complexity, it would be ideal to only use a single battery for the robot's power source. A single power source will be used for the robot, a battery holder containing 8 1.5V batteries will give us the 12V we need for the motors. Then, a step down voltage regular will be used to lower the voltage required for the other components. Fig \ref{fig:powersetup} has an overview of this.
				% !!!!!!!!!!!!!!!!!!!!!!!COME BACK TO THIS!!!!!!!!!!!!!!!!
				
				Generally cheaper alkaline batteries have about 1800 to 2800 mAh (milliamp hours) in them, so an 8 pack of these will give us about half an hour's worth of operation before the robot starts to suffer due to low power. This is a worst case scenario assuming a constant high power draw, but this should be sufficient for the purposes of prototyping and demonstration. 
			
			\subsection{Software}
			The software for the actual robot will be written in C++, compiled and deployed onto the microcontroller with the mbed SDK. At its foundation, it will be a Micro C OS with a series of configured tasks to perform various aspects of the robot's functionality. As you would expect, these tasks will come with their own memory tasks and will be designated with appropriate priorities. For the initial setup, we only need to make a skeleton OS to ensure we can properly compile and deploy code onto the microcontroller. Some relevant data structures and variables for task priorities and memory stacks, as well as a single task that prints hello world would be fine for this.
		
		\section{Designing for Requirements}
		Once this incredibly basic initial prototype is in place, we can begin to move toward truly fulfilling the project requirements. What follows is an overview of how each of the different project requirements will be met, with appropriate explanations toward the hardware and software employed.
		
			\subsection{Movement}
			The first of the three product aims outlined in the Analysis is that 'the robot must be capable of movement'.
				\subsubsection{Hardware}
				The previously mentioned dfRobot Quad Motor Driver Shield comes into play here. As shown in fig \ref{fig:shieldsetup}, the motors power and ground cables will connect to the appropriate pins on the shield. The motors can be manipulated once they receive power in this fashion. The shield is able to affect the electricity being sent to the motors by changing its polarity and by using pulse width modularity. By changing the pin to HIGH or LOW, we can affect the direction that the motor spins in (forward or backwards) and pulse width modulation allows us to affect the speed of the motor.
				
				\subsubsection{Software}
				In order to manipulate the motors we'll need some appropriate variables we can use to refer to the pins that the motor power and ground cables are connected to. %talk about pin lineup here and declaration
				
				% COME BACK TO THIS, TRIGONOMICAL NAVIGATION!

				
			\subsection{Observation}
			The second of the three product aims outlined in the Analysis is that 'the robot must be capable of observation'.
				\subsubsection{Hardware}
				Only one range finder is being used to gather the observation data, the RPLIDAR A1M8 LIDAR sensor. The sensor is composed of a platform with a motor system that spins the range scanner as it takes readings, as well as some pins that can be used for communication. Fig \ref{fig:rplidarconfig} illustrates these components.
				
				\begin{figure}[h]
					\centering
					\includegraphics[width=.9\linewidth]{SYNTHESIS/rplidar_configuration.png}
					\caption{Example shield setup}
					\label{fig:rplidarconfig}
				\end{figure}
			
				There are seven pins on the underside of the LIDAR sensor. These pins need to be connected to the appropriate microcontroller ports if the LIDAR is to work. 
				% PICTURE OF LIDAR UNDERSIDE HERE
				
				The GND pins are simple ground pins, they will need to be connected to ground pins on the microcontroller. The RX and TX pins (Recieve and Transmit respectively) are Serial pins that will be used for communication with the microcontroller. These LIDAR pins will be linked to their opposite counterparts on the microcontroller (LIDAR RX to Microcontroller TX and vice versa). The V5.0 and VMOTO are simple power pins, they will need to be connected to pins on the microcontroller that output the appropriate amount of voltage. This is how the LIDAR sensor will retrieve power. The MOTOCTL pin is the motor control pin that listens for a signal indicating that a connected device is ready to recieve data. The signal is either high (ready) or low (not ready). To make use of this pin, a generic GPIO pin from the microcontroller will be configured within the microcontroller's software and set to 1 (high) when the robot needs to begin taking scan data.
			
				To interact with the sensor, it will be connected to the robot's microcontroller. We would first need to connect the microcontroller's TX (transmit) and RX (recieve) pins to the LIDAR's inverse pins (TX to RX, RX to TX). Then, using mbed's Serial library, we can create a Serial variable supplying the microcontroller's connected RX and TX pins and parameters. From this we can invoke methods like putc() and getc(), which put and get characters into and from the serial connection respectively. In order to power the sensor, there are two pins that need to be connected. The first is LIDAR core, which is the power for the actual LIDAR scanner itself. The second is the LIDAR motor, which is what spins the scanner as it makes observations. 
				%COME BACK TO THIS AS WELL...
				
				\subsubsection{Software}
				SLAMTEC have a document detailing the LIDAR protocol. This details the specifics behind how to communicate with the LIDAR. LIDAR communication is primarily achieved through the exchange of data packets. As a basic example, in order to obtain scan measurements the host system first needs to send data packets corresponding to the begin scan command. Once the LIDAR has recieved and processed this, it should begin sending back observational data. 
				
				The format of the requests that need to be sent is documented in the LIDAR's protocol documentation\citep{rplidarprotocol}. To implement this, it would be a good idea to create a data structure in the microcontroller's C++ program with fields listed in the protocol (start flag, command, etc) and then put this field into the serial connection. A similar data structure could be populated with what is recieved from the serial channel to make it easy to process the data.
				
				SLAMTEC provide an SDK for the LIDAR sensor however, which comes in the form of header files which will automatically implement this functionality. Table \ref{table:3} has a manifest of the SDK and the functionality that it provides.
				
				\begin{table}[h!]
					\centering
					\begin{tabular}{|| l | l ||} 
						\hline
						File & Purpose \\ [0.5ex] 
						\hline
						rplidar.h  & Parent file for subsequent header files  \\ 
						rplidar\textunderscore driver.h  & Provides RPLidarDriver class for  interfacing with sensor   \\
						rplidar\textunderscore  protocol.h  & Defines structs and constants for the LIDAR protocol  \\
						rplidar\textunderscore  cmd.h & Defines request/answer structs for LIDAR protocol  \\ 
						rptypes.h & Platform independent structs and constants  \\ [1ex] 
						\hline
					\end{tabular}
					\caption{RPLIDAR SDK files}
					\label{table:3}
				\end{table}
				
				Essentially, the C++ program running on the microcontroller would need to create an RPLidarDriver variable which would be used to represent the connected LIDAR sensor. Once this is achieved the premade methods that implement the protocol functionality could be ran to achieve control over the LIDAR.
				
			\subsection{Processing Observational Data}
			The last of the three project objectives is that 'the observational data must be processed into a map'.
				\subsubsection{Hardware}
				Once the robot is receiving data from the LIDAR sensor, it needs to be transferred to an external program where the CSM software can be ran to generate a map from it. To do this a Micro SD-Card will be used. The FRDM-K64F has a Micro SD-Card socket attached to it, and the microcontroller will save the observational data to this card so that after a session of moving and scanning it can be plugged into a machine where it is able to be processed. The University's loans office can provide an 8GB card as well as an adapter allowing it to be plugged into a PC's USB slot, so this approach won't incur any extra cost. The saved data will only be angle and distance measurement pairs, so the card's (relatively) small size won't pose any memory issues. 
				
				\subsubsection{Software}
		

		
	\chapter{Implementation}
		\section{Introduction}
		\section{Initial Prototype}
			\subsection{Hardware}
			\subsection{Software}
		\section{Implementing Requirements}
			\subsection{Movement}
				\subsubsection{Hardware}
				\subsubsection{Software}
			\subsection{Observation}
			Implementation of the robot's ability to take observations about its environment first involved a hardware configuration. The LIDAR sensor needed to be connected to the microcontroller so that it could draw power from the battery, as well as communicate with the microcontroller allowing it to receive commands and send observational data. Once the hardware connection was established, software on the microcontroller had to be capable of operating the LIDAR sensor and recieving its observations.
				\subsubsection{Hardware}
				As previously discussed, the seven pins on the underside of the LIDAR sensor need to be correctly connected up to the appropriate microcontroller pins if the sensor is to be used.
				
				Table \ref{table:4} gives an overview of which LIDAR pins needed to be connected to which microcontroller pins.
				
				\begin{table}[h!]
					\centering
					\begin{tabular}{|| l | l ||} 
						\hline
						LIDAR Pin & K64F Pin \\ [0.5ex] 
						\hline
						GND  & GND  \\ 
						RX  & TX   \\
						TX  & RX  \\
						V5.0 & 5v  \\ 
						GND & GND  \\ 
						MOTOCTL & D7  \\ 
						VMOTO & 3v3  \\ [1ex] 
						\hline
					\end{tabular}
					\caption{RPLIDAR SDK files}
					\label{table:3}
				\end{table}
			
				Whilst most of these pins are straight forward, one that needed some additional configuration was the D7 pin used for the MOTOCTL (motor control). As previously discussed in the design, the MOTOCL pin will listen for either a high or a low signal. If the signal is high, the LIDAR has the okay to begin scanning and outputting data. If the signal is low, even if the LIDAR has power and recieves the appropriate command it will not function as the connected device has not given the ready signal. The K64F features numerous GPIO pins that can be easily configured for usage in situations like this. One such pin (D7) was simply declared as a basic DigitalOut allowing the microcontroller to manipulate the pin's polarity.
				
				\begin{lstlisting}
				DigitalOut dtr(D7);
				\end{lstlisting}
				
				When we want the sensor to output data this pin can be set to high....
				\begin{lstlisting}
				dtr = 1;
				\end{lstlisting}
				
				...or low.
				\begin{lstlisting}
				dtr = 0;
				\end{lstlisting}
				
				
				\subsubsection{Software}
				
				
				
			\subsection{Processing Observational Data}
				\subsubsection{Hardware}
				\subsubsection{Software}
	
	
	\chapter{Testing}
		\section{Introduction}
		\section{Initial Prototype}
			\subsection{Hardware}
			\subsection{Software}
		\section{Implementing Requirements}
			\subsection{Movement}
				\subsubsection{Hardware}
				\subsubsection{Software}
			\subsection{Observation}
				\subsubsection{Hardware}
				\subsubsection{Software}
			\subsection{Processing Observational Data}
				\subsubsection{Hardware}
				\subsubsection{Software}