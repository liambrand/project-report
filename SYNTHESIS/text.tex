\part{Synthesis}
	\chapter{Introduction}
	Now that the relevant product fields are understood to a sufficient degree and the product's needs have been established we can begin to plan out how we will achieve the project's objectives. First we will established the methological approach that will structure the product's development. Then, for each of the core product requirements we will look at a high level design overview of how the requirement will be met, the actual implementation of this design, and testing to evaluate how well this requirement has been satified by this implementation.
	
	\chapter{Methological Approach}
	The project employed a prototyping methodology for development. This involves the system requirements being outlined and understood with an initial design being created from these requirements. Following this the first prototype is built, which often resembles an incredibly basic scaled down version of how the final system should ideally look. This prototype is then thoroughly evaluated, with potential changes that would bring the prototype closer to the final system being figured out. Finally, these changes are then implemented and the prototype is evaluated once again. This is repeated until the product has reached its ultimate goals.
	
	%prototyping diagram?
	
	This was employed partially due to the hardware nature of the product. Key features of the robot hinged on aspects like movement and observation being possible, so it was important to get a prototype that has this basic functionality up and working and quickly as possible. The prototyping approach allowed a pure focus on getting these functions working, and once they were the robot could be improved iteratively.
	
	\chapter{Design}
	
	
	Before any functional implementation to satisfy the product requirements can take place, a thorough understanding of what needs to actually be implemented and how it is to be implemented needs to be achieved. In the design section, an overview of the proposed implementation for each of the product requirements will be discussed. 
	
	Given that the project is employing the prototyping, the proposed solutions will be broken down into basic steps. These steps will form the basis of the prototype iterations, with each subsequent prototype ideally achieving an extra step of functionality over the previous prototype. This will allow easily measurable progress during the project's development which will help determine whether or not it is on track, as well as ensuring each developed prototype has well defined goals that are not too small or great in scope.
	
		\subsection{Movement}
			\subsubsection{Hardware}
			As described previously in the Analysis, the foundation of the robot's structure is the three wheeled omnidirectional chassis. This will provide some of the fundamentals needed to achieve a moving chassis, chiefly the chassis itself as well as motors to drive the omni directional wheels, as well as a structure which can house the other components such as the power supply and the microcontroller which will dictate the motors.
			
			\subsubsection{Software}
		
		
		\subsection{Observation}
			\subsubsection{Hardware}
			\subsubsection{Software}
		
		\subsection{Processing Observational Data}
		
		
	\chapter{Implementation}
		\subsection{Movement}
		\subsection{Observation}
		\subsection{Processing Observational Data}
		
	\chapter{Testing}
		\subsection{Movement}
		\subsection{Observation}
		\subsection{Processing Observational Data}