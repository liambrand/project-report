\part{Synthesis}
	\chapter{Introduction}
	Now that the relevant product fields are understood to a sufficient degree and the product's needs have been established we can begin to plan out how we will achieve the project's objectives. First we will established the methological approach that will structure the product's development. We will then look into the basic design that is needed to achieve the different product requirements. This will allow the actual implementation to begin, and once that is complete appropriate tests will be taken out to ensure that the needs have been properly met.
	
	\chapter{Methological Approach}
	
	\chapter{Design}
		\subsection{Movement}
			\subsubsection{Hardware}
			As described previously in the Analysis, the foundation of the robot's structure is the three wheeled omnidirectional chassis. This will provide some of the fundamentals needed to achieve a moving chassis, chiefly the chassis itself as well as motors to drive the omni directional wheels, as well as a structure which can house the other components such as the power supply and the microcontroller which will dictate the motors.
			
			\subsubsection{Software}
		
		
		\subsection{Observation}
		\subsection{Processing Observational Data}
		
	\chapter{Implementation}
		\subsection{Movement}
		\subsection{Observation}
		\subsection{Processing Observational Data}
		
	\chapter{Testing}
		\subsection{Movement}
		\subsection{Observation}
		\subsection{Processing Observational Data}