\chapter{Abstract}
This report details the research, planning and development carried out in order to create a ground-based autonomous robot capable of tracking its own location whilst mapping an environment. It begins with a brief introduction to the project's background and subject area, followed by a brief explanation of the project's motivation, discussing some of the problems that can be addressed by such a robot. The project's main objectives are outlined as well as the produced work and the core stages of the project.

An analysis was conducted to gain a greater understanding of relevant project fields, as well as to understand what the potential solutions were to the problems so that accurate requirements could be outlined. Using what had been learned about the key fields and the established requirements, different tools were compared and chosen for the project based on factors like efficiency, cost and ease of use.

The project's synthesis begins with a high level design overview outlining the project's methodology and plan for the implementation to create the product's intended functionality. A description of the implementation is then provided detailing what happened as the proposed design was followed. This is followed by a discussion on the product's testing which was used to determine how well the product had satisfied its initial requirements.

Finally the product and the process used throughout the project was evaluated, with a discussion of the strengths and weaknesses of what was done. The ultimate conclusion was that the product was a failure based on its failing of numerous vital tests, and that the process which was used to address the project was heavily flawed. Conclusions were drawn from this evaluation and suggestions were made for further work both with the product and in the wider field the project's technology resides in.




